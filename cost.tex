\section{Cost Breakdown}

At Viridity Capital LLC, we strive to be very efficient with our costs, and
have already cut costs upfront in this proposal.

The costs here are indicated for a year, and as there is options for more years
on the contract, our cost for each option year will be the same as the first.

\subsection{Direct Labor}

We have calculated the 2000 hrs worked as 2080 hrs minus 10 Federal holidays
for the year.

It is noted that the CEO has significantly more responsibilities
than the engineers, as he has to meet with stakeholders, provide roadmap,
planning, and also is an engineer as well.

\renewcommand{\arraystretch}{1.2}
\begin{center}
  \begin{tabular}{|l|l|c|r|r|}
    \hline
    \tb{Name}   & \tb{Title}      & \tb{\# Hours Worked} & \tb{Hourly Rate} & \tb{Total Cost} \\\hline
    Michael You & CEO \& Engineer & 2000                 & \$75.00          & \$150,000       \\\hline
    Joseph Kim  & Engineer        & 2000                 & \$37.50          & \$75,000        \\\hline
    Wilson Yu   & Engineer        & 2000                 & \$37.50          & \$75,000        \\\hline
    Emily Cheng & Engineer        & 2000                 & \$37.50          & \$75,000        \\\hline
    Lin Xie     & Engineer        & 2000                 & \$37.50          & \$75,000        \\\hline
                &                 &                      & \tb{Total}       & \$450,000       \\\hline
  \end{tabular}
\end{center}

\subsection{Indirect Costs}

We will need to purchase compute for hosting the website, storing data in
databases, and providing analytics. We will be using a combination of AWS
Lambdas and Elastic Beanstalk. We have estimated \$10,000 a month for the
compute costs as we have seen similarly scaled businesses use around that much
for their services which provide websites with analytics for users worldwide.

We will save costs by having employees use our own laptops to work
on. The company has agreed that we don't need to provide benefits to the
employees as well to additionally save costs. We will also choose to forgo
purchasing subscriptions for coworking space such as WeWork.

\begin{table}[H]
  \centering
  \begin{tabular}{|l|c|r|r|}
    \hline
    \tb{Item}                                                                                                          & \tb{Quantity}     & \tb{Unit Cost}      & \tb{Sum Cost} \\\hline
    Compute (AWS)\tablefootnote{Calculated using \href{https://calculator.aws/}{https://calculator.aws/}}              & 12                & \$10,000/month      & \$120,000     \\\hline
    Statista\tablefootnote{\href{https://ask.statista.com/pricing/}{Find pricing here.} Tax added for completeness.}   & 1                 & \$20,000            & \$20,000      \\\hline
    GitHub Pro\tablefootnote{\href{https://github.com/pricing}{Enterprise plan}}                                       & $12 \cdot 5 = 60$ & \$21/month per user & \$1,260       \\\hline
    MongoDB\tablefootnote{\href{https://www.mongodb.com/pricing}{Atlas Dedicated, 2 DBs for the data and the surveys}} & 12                & \$120/month         & \$1440        \\\hline
                                                                                                                       &                   & \tb{Total}          & \$142,700     \\\hline
  \end{tabular}
\end{table}


\clearpage
\subsection{Travel}

We will have travel taken by all members once a month to meet in person. As we
have workers in New York City and Washington DC, we will travel by train. The
Per Diem cost includes hotel and food for the day.

\begin{table}[H]
  \centering
  \begin{tabular}{|l|c|r|r|}
    \hline
    \tb{Item}             & \tb{Quantity}     & \tb{Unit Cost} & \tb{Sum Cost} \\\hline
    Train                 & $12 \cdot 5 = 60$ & \$200          & \$12,000      \\\hline
    Ground Transportation & $12 \cdot 5 = 60$ & \$200          & \$12,000      \\\hline
    Per Diem              & $12 \cdot 5 = 60$ & \$300          & \$18,000      \\\hline
                          &                   & \tb{Total}     & \$42,000      \\\hline
  \end{tabular}
\end{table}

\subsection{Total Costs}
Summing up our total costs, we arrive at the final cost of \totalCost.

\renewcommand{\arraystretch}{1.2}
\begin{table}[H]
  \centering
  \begin{tabular}{|l|r|}
    \hline
    \tb{Category}       & \tb{Total Cost} \\\hline
    \tb{Direct Labor}   & \$450,000       \\\hline
    \tb{Indirect Costs} & \$142,700       \\\hline
    \tb{Travel}         & \$42,000        \\\hline
    \tb{TOTAL}          & \totalCost      \\\hline
  \end{tabular}
\end{table}

As mentioned earlier, the final cost of \totalCost is for one year. This cost is the same for each option year.